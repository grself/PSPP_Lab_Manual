\section{Miscellaneous Notes}
% the text width in a normal line is 11.80737cm
% To make a TODO note, start a comment with two or more consecutive capital letters.

\section{Captions and Labels}
%  All lables are the same as the captions, but lowercase and no punctuation. 
%  Also, all spaces are filled with underscores.
%  Occasionally, labels may also need a bit of verbiage to differentiate 
%  them from each other, for example, there will be lots of ``example'' 
%  sections. In that case, put additional verbiage at the end of the label.
%  Labels are referenced book-wide, not just chapter-wide. So prefix 
%  the 2-digit chapter number at the start of the label (Appendicies 
%  will use ``ap''). Also, use the following label designations:
%	ch: -- chapter
%	sec: -- section
%	subsec: -- sub-section
%	para: -- paragraph
%	tab: -- table
%	eq: -- equation
%	fig: -- figure
%	lst: -- code listing
%	soln: -- the solution to an equation
%	A full label would look something like: \label{03:tab:truth_table_for_or}
%  The number of degrees for label position within an image node are:
%	E=0
%	NE=45
%	N=90
%	NW=135
%	W=180
%	SW=225
%	S=270
%	SE=315

\section{My Macros}
Shft+F1: List Inline

%************************************************************
\section{acronyms}
% Define a new acronym in "Contents.tex"
% The definition takes the form: \acro{POS}{Product of Sums}
% Enter them in alphabetic order since they will be printed in the Table of Acronymns 
% 	in the same order that they are listed on the Contents page
%***************************************************************
\ac{ACR}   % prints the acronym text and then the acronym in brackets 
\acl{ACR}  % prints the long version (the definition without acronym)
\acf{ACR}  % prints the full name of the acronym plus the acronym
\acs{ACR}  % prints the short version (no definition)
\acp{ACR}  % prints the plural version
\acfp{ACR} % prints the full version in plural
\aclp{ACR} % prints the long version in plural

%*****************************************************
\section{Truth Table}
%*****************************************************
\begin{table}[H]
	\sffamily
	\newcommand{\head}[1]{\textcolor{white}{\textbf{#1}}}		
	\begin{center}
		\rowcolors{2}{gray!10}{white} % Color every other line a light gray
		\begin{tabular}{ccc} 
			\rowcolor{black!75}
			\multicolumn{2}{c}{\head{Inputs}} & \head{Output} \\
			A & B & Y \\
			\hline
			0 & 0 & 0 \\
			0 & 1 & 1 \\
			1 & 0 & 1 \\
			1 & 1 & 1 
		\end{tabular}
	\end{center}
	\caption{Truth Table for OR}
	\label{03:tab:truth_table_for_or}
\end{table}


%******************************************************
\section{3-Line Equation}
%******************************************************
\begin{align}
	\label{03:eq:identity_example}
	1101_2 &= 123 \\
	\nonumber
	&= 123 \\
	\nonumber
	&= 123
\end{align}


\section{Solving an Equation Step-by-step}
\begin{align}
	\label{04:soln:solving_equation_one}
	AB+BC(B+C) && \text{Original Expression} \\
	\nonumber
	AB+BBC+BCC && \text{Distribute BC} \\
	\nonumber
	AB+BC+BC && \text{Idempotence: BB=B and CC=C} \\
	\nonumber
	AB+BC && \text{Idempotence: BC+BC=BC} \\
	\nonumber
	B(A+C) && \text{Factor} \\
\end{align}

%**************************************************************************
\section{Margin Paragraph} 
%**************************************************************************
% Use to create a sidebar paragraph out in the margin
\marginpar{Whatever.}

%***************************************************************************
\section{Text Box}
%***************************************************************************
% Creates a nice boxed text with a title and main section
\begin{tcolorbox}[colback=blue!5!white,colframe=blue!75!black]
	% Upper half of box: my "title" area
	\textcolor{blue}{\textbf{Interesing Note}}
	% Lower half of the box: the content
	\tcblower
	Whatever.
\end{tcolorbox}

%***************************************************************************
\section{ToDo}
%***************************************************************************
% Create a ToDo note in the text. This also creates a new clickable TODO
% section in the ``structure'' box on the left side of the page.
%TODO This is a todo note.

%***************************************************************************
\section{Code Snip}
%***************************************************************************
\lstset{ %
  caption={caption},
  label=SL:lst:listing01,
  numbers=left,
  language=Verilog
}
\begin{lstlisting}
  Code here
\end{lstlisting}

%***************************************************************************
\section{Code From File For Text Area}
%***************************************************************************
\lstset{ %
  caption={add2 Solution},
  label=L03:lst:add2,
  basicstyle={},
  upquote=false, 
  numbers=left,
  language=Verilog
}
\lstinputlisting[firstline=7]{ckt/lab03.add2.v}

\section{Code From File For Code Snip}
%***************************************************************************
\lstset{ %
  caption={add2 Solution},
  label=L03:lst:add2,
  basicstyle={\ttfamily},
  upquote=true,
  numbers=none,
  language=teX
}
\lstinputlisting[firstline=7]{ckt/lab03.add2.v}

%***************************************************************************
\section{List In Line}
%***************************************************************************
\lstinline[columns=fixed]|code|

%***************************************************************************
\section{Wrap Figure}
%**************************************************************************
\begin{wrapfigure}{O}{0.2\textwidth}
	\caption{} % No text, wraps badly in very narrow space (does print fig number)
	\label{BM:fig:gray_code_disc} 
	\centering
	\includegraphics[width=0.2\textwidth]{gfx/gray_code_disc} 
\end{wrapfigure}


%***************************************************************************
\section{Regular Figure}
%**************************************************************************
\begin{figure}[H]
  \centering
  \frame{\includegraphics{ckt/lab_hello.png}}
  \caption{Executing ``Hello World''}
  \label{lab01:fig:executing_hello_world}
\end{figure}

%***************************************************************************
\section{Style Guide}
%**************************************************************************
All instances of the term PSPP should be formatted as \acs{PSPP}

For menu selections:
Click \textsc{\fbox{Analyze $ \rightarrow $ Descriptive Statistics $ \rightarrow $ Frequencies}}

All dataset and variable names should be italicized

Names of elements in windows, like text boxes, should be in normal font but surrounded by quotes