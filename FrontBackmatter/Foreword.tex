%*****************************************
\chapter*{Forward}\label{for:forward}
%*****************************************

I have taught BASV 316, \textit{Introductory Methods of Analysis}, online for the University of Arizona in Sierra Vista since 2010 and enjoy working with students on research methodology. From the start, I wanted students to work with statistics that are commonly found in research. It is my belief that the best way to understand what statistics can, and cannot, prove is to calculate values using a known dataset. As I evaluated statistical software for this class I had three criteria:

\begin{itemize}
  \item \textbf{Open Educational Resource (OER)}. It is important to me that students use software that is available free of charge and is supported by the entire web community. 
  \item \textbf{Platform}. While most of my students use a Windows-based system, some use Macintosh and it was important to me to use software that is available for all of those platforms. As a bonus, most OER software is also available for the Linux system, though I'm not aware of any of my students who are using Linux.
  \item \textbf{Longevity}. I wanted a system that could be used in other college classes or in a business setting after graduation. That way, any time a student spends learning the software in my class will be an investment that can yield results for many years.
\end{itemize}

I originally wrote a series of six lab exercises (later expanded to nine) using R-Project since that software met these three criteria. Moreover, R-Project is a recognized standard for statistical analysis and could be easily used for even peer-reviewed published papers. Unfortunately, I found R-Project to be confusing to students since it is text-based with rather complex commands. I found that I spent a lot of time just teaching students how to set up a single test with R-Project instead of analyzing the result. In the spring of 2017 I changed to \texttt{SOFA} (\textit{Statistics Open For All}) because it is much easier to use and still met my criteria. However, SOFA is a stand-alone product that students would not likely be able to use beyond this class so in the fall of 2017 I changed to \texttt{PSPP}. \texttt{PSPP} looks and works like SPSS, which is the leading statistical analysis software package used in research around the world. \texttt{PSPP} is a simplified version of SPSS and has only a few statistical analysis options available, but those are enough for many undergraduate projects. Moreover, I believe that students who learn \texttt{PSPP} in this class and later need the power of SPSS will find the transition to be very smooth since the two programs have a similar menu structure.

This lab manual explores many aspects of \texttt{PSPP} but does not attempt to dig into every corner of this software. It is my hope that students will find the labs instructive and will then be able to use \texttt{PSPP}, or smoothly transition into SPSS, for other classes. This lab manual is published under a Creative Commons license with a goal that other instructors will modify it to meet their own needs. I always welcome comments and will improve this manual as I receive feedback.

\bigskip
\begin{flushright}
  \textemdash  George Self
\end{flushright}


